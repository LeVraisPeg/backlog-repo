\documentclass{article}
\usepackage{amsmath, amssymb, geometry}
\usepackage{tikz}
\usepackage{titlesec}
\usepackage{lmodern}
\usepackage[french]{babel}
\newcommand{\HRule}{\rule{\linewidth}{0.5mm}}
\geometry{a4paper, margin=2.5cm}

\title{\Huge\bfseries\vspace{2cm}Résumé synthétique\\[0.5em] \LARGE Modélisation géométrique et Maillage}
\author{\Large Christian Nguyen \ Département d'informatique -- Université de Toulon}
\date{\vspace{1cm} \today}

\begin{document}

\begin{titlepage}
\centering
\begin{tikzpicture}[remember picture, overlay]
\node[opacity=0.1] at (current page.center) {\includegraphics[width=\paperwidth,height=\paperheight,keepaspectratio]{image.png}};
\end{tikzpicture}

\vspace*{2cm}

% Titre principal
{\Huge\bfseries Compte Rendu de Projet\\[0.5em] \LARGE SAE821 -- Gérer un projet}

\vspace{1.5cm}

% Ligne de séparation
\HRule
\vspace{1cm}

% Informations sur les auteurs
\Large{Rapport de présenation de la modélisation d'un système de recommandation  respectant la vie privée}\\[0.5em]

\vspace{1cm}
\HRule\\[11cm]
    \begin{flushleft}
        \small
        \textbf{Pegliasco Matteo}\\
        \textbf{Berge Enzo}\\
        \textbf{Audouard Florian}\\
        \textbf{Hermelin Lois}\\
        \textbf{Master Informatique et Mathématiques}\\
        Université de Toulon, La Garde, Var, France
    \end{flushleft}
    \vfill

\end{titlepage}


\tableofcontents
\newpage

\section*{Introduction}
Ce rapport technique est un comtpe rendu dans le cadre de la SAE821 dont l'objectif est la conception et la modélisation d’un système de recommandation respectueux de la vie privée des utilisateurs. Les systèmes de recommandation sont aujourd’hui omniprésents dans de nombreux domaines, notamment sur les plateformes de commerce en ligne, les réseaux sociaux ou encore les services de streaming. Ils permettent d’orienter les utilisateurs vers des contenus pertinents en se basant sur l’analyse de leurs comportements et préférences.
Cependant, pour la majorité, ce traitement massif de données personnelles soulèvent des problématiques importantes liées à la protection de la vie privée, l'intégritée et la sécurité des données. Il est donc essentiel de réfléchir à des solutions techniques permettant de préserver la confidentialité des données tout en maintenant l’efficacité des recommandations.
L’objectif de ce projet est de modéliser un système de recommandation de films prenant en compte tout ces critères, en utilisant des méthodes préservant la protection des données utilisateurs. Ce rapport présentera les différentes étapes de la modélisation, les choix techniques réalisés, ainsi que les méthodes mises en œuvre pour concilier personnalisation et respect de la vie privée.

\section{Recherches théoriques}
\subsection{Systèmes de recommandation}
\subsubsection{Deep learning}
\subsubsection{Collaboratibe filtering}
\subsubsection{Matrix factorization}
\subsubsection{Chaine de Markov}
\subsection{Confidentialité}
\subsubsection{Homomorphic encryption}
$ $\\
Le chiffrement homomorphe est une technique de cryptographie qui permet d'effectuer des calculs sur des données chiffrées sans avoir besoin de les déchiffrer au préalable mais dont le résultat déchiffré conserve les opérations entre les données claires.
Pour cela, plusieurs catégories de chiffrements existent:\\
\begin{itemize}
    \item \textbf{Chiffrement partiellement homomorphe}: Il permet de réaliser des opérations sur des données chiffrées pour une loi spécifique, par exemple l'addition ou la multiplication entre des données chiffrées.
    \item \textbf{Chiffrement presque entièrement homomorphe}: Il permet de réaliser un nombres limité d'opérations sur des données chiffrées pour plusieurs lois spécifiques, par exemple l'addition et la multiplication entre des données chiffrées, à cause d'une accumulation de bruit.
    \item \textbf{Chiffrement entièrement homomorphe}: Il permet de réaliser un nombre illimité d'opérations sur des données chiffrées pour toutes les lois possibles, mais il est limité par la taille du bruit accumulé lors des opérations.
\end{itemize}
$ $\\
Dans le cadre de ce projet nous nous sommes intéressés particulièrement au chiffrement entièrement homomorphe pour sa flexibilité d'utilisation pour différents système de recommandations.\\
Ainsi, plusieurs approches existent pour la mise en place du chiffreemnt homomorphe: \\
\begin{itemize}
    \item \textbf{Chiffrement DGHV}: Le chiffrement DGHV est un système de chiffrement homomorphe, opérant sur des entiers, basé sur le problème de la somme de sous-ensembles clairsemée et le problème du PGCD approché.
    \item \textbf{Chiffrement GSW avec bootstrapping}: Le chiffrement GSW est un système de chiffrement homomorphe, opérant sur des bits, basé sur le problème de la factorisation des entiers et le problème du logarithme discret.
    \item \textbf{Chiffrement TFHE}: Le chiffrement TFHE est un système de chiffrement homomorphe, opérant sur des bits, basé sur la version torique du problème Learning With Errors.
\end{itemize}

\subsubsection{Differential privacy}
$ $\\
La confidentialité différentielle est une approche qui vise à garantir la protection de la vie privée des individus dans les ensembles de données en restreignant la relation entre les données de sortie et les données de l'utilisateur.
Pour cela, plusieurs approches existent:\\
\begin{itemize}
\item \textbf{Mécanisme exponentiel}: Le mécanisme exponentiel permet de sélectionner aléatoirement un élément parmi un ensemble de données, en favorisant les éléments ayant le plus de poids selon une fonction d'évaluation donnée.
\item \textbf{Réponse aléatoire}: Le mécanisme de réponse aléatoire vise à brouiller les réponses données en remplaçant aléatoirement certaines réponses par des valeurs aléatoires.
En connaissant la probabilité d'obtenir une valeur aléatoire, il est alors possible d'estimer la proportion de données réelles ainsi que la variance au sein des données.
\item \textbf{\emph{K-anonymisation} et \emph{L-diversité}}: La \emph{K-anonymisation} et la \emph{L-diversité} consistent à masquer les éléments de quasi-identification d'un
 individu en le regroupant avec $K-1$ autres individus dans une représentations plus large, tout en conservant la visibilité des $L$ informations spécifiques à chaque utilisateurs et destinées à être partagées.
\item \textbf{Mécanisme de Laplace ou Gaussien}: Le mécanisme de Laplace ou Gaussien consiste à altérer les données des utilisateurs pour masquer les données réelles tout en conservant la répartition moyenne des données en ajoutant un bruit aléatoire.
\end{itemize}
\section{Démarche choisie}
\subsection{Motivations du choix}
\subsubsection{Confidentialité des données}
$ $\\
Le cadre de ce projet est de mettre en place un système de recommandation de films respectant la vie privée des utilisateurs. Nous avons
considéré que la base de données des utilisateurs ainsi que les films proposés pouvaient évoluer au fil du temps.
Pour la protection de la vie privée des utilisateurs, nous avions considéré l'utilisation du chiffrement homomorphe TFHE, mais
la complexité de son implémentation et de son utilisation, avec notamment l'interaction entre différents langages de programmation (JAVA, C++, RUST),
ainsi que la complexité de l'utilisation de l'apprentissage machine avec une bibliothèque spécialisée, nous a poussés à nous intéresser davantage à la confidentialité différentielle.
Plus précisément, nous nous sommes particulièrement intéressés à la mise en place du mécanisme de bruitage et du mécanisme exponentiel pour la mise
en place de la confidentialité différentielle pour, respectivement, les recommandations pour un utilisateur et les recommandations personnalisées
d'un utilisateur ayant déjà évalué une certaine quantité de films.\
Le mécanisme exponentiel est destiné à la recommandation à froid en sélectionnant aléatoirement un film recommandé afin de limiter les informations
que l'utilisateur peut tirer des habitudes des autres utilisateurs, tout en favorisant les films qui n'ont pas encore été regardés ou les films avec de bonnes notes moyennes.
En ce qui concerne le mécanisme de bruitage, nous avons choisi d'utiliser le mécanisme de Laplace pour masquer les données des utilisateurs.\
Ce choix est motivé par le contexte de ce projet. En effet, les utilisateurs peuvent être créés ou supprimés. Ainsi, et contrairement au mécanisme gaussien,
le mécanisme de Laplace est un mécanisme $\epsilon$-indistinguable, c'est-à-dire qu'il garantit que l'absence ou la présence d'un
utilisateur impacte la distribution des données de manière limitée. En effet, si l'on considère $D_1$ et $D_2$ deux ensembles de données
de cardinalité distincte de 1, $A$ un algorithme aléatoire $\epsilon$-indistinguable et $S \subset A$, alors on a par définition :
\begin{equation}
P(A(D_1) \in S) \leq P(A(D_2) \in S) e^{\epsilon}
\end{equation}
Ici, en considérant $A$ comme la fonction qui ajoute du bruit et qui somme les notes des utilisateurs pour un film, et $S$ un intervalle de valeurs possibles,
en ajoutant une note à un film, la probabilité que la somme des notes d'un film soit dans le même intervalle $S$ est limitée par le facteur $e^{\epsilon}$.
Le mécanisme gaussien, quant à lui, est un mécanisme $(\epsilon,\delta)$-differential privacy, c'est-à-dire qu'il admet une marge d'erreur $\delta$ générée
par la concentration des valeurs de la distribution gaussienne autour de 0, limitant ainsi la génération de bruit à forte valeur.\\
$ $\\
Enfin, et comme présenté précédemment, nous avons considéré l'utilisation du mécanisme exponentiel pour la recommandation à froid d'un utilisateur.
Cette approche permet à la fois de favoriser certains films grâce à une fonction d'utilité définie dans le système, tout en limitant le risque que 
l'utilisateur puisse extraire des informations individuelles sur les autres utilisateurs ayant permis de générer les recommandations avec une bonne 
calibration du parmètre $\epsilon$.
\subsubsection{Models de recommandations}


\subsection{Implémentation dans le système}
\subsection{Limites et perspectives}


\section{Résultats expérimentaux}
\subsection{Précision des recommandations}
\subsection{Performances computationnelles}
\subsection{Mesure de la confidentialité}
\subsection{Résumé}

\section*{Références}
\addcontentsline{toc}{section}{Références}

\appendix
\section*{Annexes}
\addcontentsline{toc}{section}{Annexes}
Données supplémentaires, résultats détaillés, ou code pertinent.

\end{document}
