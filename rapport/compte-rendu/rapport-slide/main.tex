\documentclass{beamer}

\usetheme{metropolis}
\usepackage[french]{babel}
\usepackage[utf8]{inputenc}
\usepackage[T1]{fontenc}
\usepackage{graphicx}
\usepackage{tikz}
\newcommand{\roundimage}[2][2cm]{%
  \begin{tikzpicture}
    \clip[rounded corners=1cm] (0,0) rectangle (#1,#1);
    \node[anchor=south west, inner sep=0] at (0,0) {\includegraphics[width=#1]{#2}};
  \end{tikzpicture}%
}
\usepackage{pifont} % À mettre dans le préambule

% Icônes utiles
\newcommand{\question}{\ding{72}}   % ❓
\newcommand{\blocked}{\ding{55}}    % 🚫
\newcommand{\lock}{\ding{115}}      % 🔐
\newcommand{\target}{\ding{93}}     % 🎯


\setbeamertemplate{title page}{
    \vbox{}
    \vfill
    \begin{center}
        {\small % ⬅️ Réduction ici
        
        % Logos en haut
        \begin{minipage}{0.45\textwidth}
            \flushleft
            \includegraphics[height=1.2cm]{logo}
        \end{minipage}%

        \vspace{0.5cm}

        {\usebeamerfont{title}\Huge \inserttitle\par}
        \vspace{0.3cm}
        {\usebeamerfont{subtitle}\Large \insertsubtitle\par}
        \vspace{1cm}
        {\usebeamerfont{institute}\normalsize \insertinstitute\par \usebeamerfont{date}\insertdate\par}
        
        } % fin du \small
    \end{center}
    \vfill
}

\setbeamertemplate{section in toc}[sections numbered]


% Informations
\title[Soutenance de Projet]{Soutenance de Projet}
\subtitle{SAE-821 Développement d’un
moteur de recommandations
respectueux de la vie privée}
\author{
    \textbf{Enzo Berge, Lois Hermelin}\\
    \textbf{Florian Audouard, Mattéo Pegliasco}
}
\institute{Université de Toulon, La Garde}
\date{2 Juin 2025}

\begin{document}

% Page de titre
\begin{frame}
    \titlepage
\end{frame}

% Sommaire
\begin{frame}
    \frametitle{Sommaire}
    \tableofcontents
\end{frame}

\section{Introduction}
\begin{frame}
    \frametitle{Introduction}
    \small
    Les moteurs de recommandation sont devenus omniprésents dans notre quotidien : sur les plateformes de streaming, 
    les sites de e-commerce ou encore les réseaux sociaux. Leur but est de suggérer à l’utilisateur du contenu pertinent 
    en se basant sur ses préférences, son historique de navigation ou les comportements d’utilisateurs similaires.

    Pour atteindre cette efficacité, ces systèmes s’appuient sur l’analyse de vastes quantités de données personnelles. 
    Cette collecte, souvent invisible pour l’utilisateur, soulève de nombreuses préoccupations liées à la confidentialité 
    et à la protection de la vie privée.

    Dans le cadre de ce projet, nous nous sommes donné pour objectif de concevoir un moteur de recommandations simple mais performant, 
    capable de respecter les données personnelles des utilisateurs, en explorant des approches innovantes de préservation de la vie privée.
\end{frame}



% Problématique
\section{Problématique}
\begin{frame}
    \frametitle{Problématique}

    \small % Réduit la taille du texte

    \textbf{\question~Problème central :} 
    \textit{Comment proposer des recommandations pertinentes sans compromettre la vie privée des utilisateurs ?}

    \vspace{0.2cm}
    \textbf{\blocked~Limites actuelles :}
    \begin{itemize}
        \item Solutions \textbf{centralisées} : profilage, tracking, fuites de données.
        \item Collecte massive de données personnelles.
    \end{itemize}

    \textbf{\lock~Deux approches de protection :}
    \begin{itemize}
        \item \textbf{Differential Privacy} : bruit pour anonymiser.
        \item \textbf{Cryptographie homomorphe} : calcul sur données chiffrées.
    \end{itemize}

    \textbf{\target~Objectif :} 
    Développer un moteur de recommandation basé sur \texttt{MovieLens}, tout en garantissant la confidentialité.
\end{frame}

% User Story
\section{Conception UML}
\begin{frame}
    \frametitle{User Story}
    \small
    Le point de départ de notre projet repose sur une vision centrée utilisateur. La user story suivante résume notre objectif principal :
    \begin{block}
        \textit{« En tant qu’utilisateur, je veux recevoir des recommandations personnalisées sans que mes données personnelles soient stockées en ligne. »}\\
        \textit{« En tant qu’utilisateur, je veux pouvoir consulter mes avis sur les films que j’ai vus, sans que mes données personnelles soient stockées en ligne. »}\\
        \textit{« En tant qu’utilisateur, je veux pouvoir rechercher des films par titre/genres, sans que mes données personnelles soient stockées en ligne. »}
    \end{block}
    Cette phrase illustre une exigence fondamentale : proposer un service personnalisé tout en assurant une protection forte de la vie privée.
\end{frame}

\begin{frame}
    \frametitle{Diagramme de cas d'utilisation}
        \vspace{0.5cm}
    \begin{center}
        \includegraphics[width=0.75\textwidth]{uc_utilisateur.png}
        
        {\small Diagramme de cas d'utilisation de l'utilisateur}
    \end{center}
\end{frame}

\begin{frame}
    \frametitle{Diagramme de classes}
        \vspace{0.5cm}
    \begin{center}
        \includegraphics[width=0.75\textwidth]{uc_utilisateur.png}
        
        {\small Diagramme de classes du système}
    \end{center}
\end{frame}

\begin{frame}
    \frametitle{Diagramme de deploiement}
        \vspace{0.5cm}
    \begin{center}
        \includegraphics[width=0.75\textwidth]{uc_utilisateur.png}
        
        {\small Diagramme de deploiement du système}
    \end{center}
\end{frame}


% Fonctionnalités - Vue générale
\section{Fonctionnalités}
\begin{frame}
    \frametitle{Fonctionnalités - Vue générale}
    \begin{itemize}
        \item Recommandations pertinentes
        \item Pas de collecte centralisée
        \item Interface intuitive
        \item Système modulaire
    \end{itemize}
\end{frame}

% Fonctionnalités - Techniques
\begin{frame}
    \frametitle{Fonctionnalités - Techniques}
    \begin{itemize}
        \item Algorithmes de filtrage collaboratif
        \item Traitement local des données utilisateur
        \item Anonymisation des échanges entre profils
        \item Visualisation des recommandations avec Streamlit
    \end{itemize}
\end{frame}

% Architecture - Diagramme UML
\section{Architecture}
\begin{frame}
    \frametitle{Architecture du système}
    \begin{center}
        \includegraphics[width=0.8\linewidth]{logo}
        % Remplacez "uml_diagram" par le nom réel de votre image UML
    \end{center}
\end{frame}

% Architecture - Explication
\begin{frame}
    \frametitle{Modules principaux}
    \begin{itemize}
        \item \textbf{Interface utilisateur} : Streamlit pour l’interaction
        \item \textbf{Moteur de recommandation} : filtrage collaboratif local
        \item \textbf{Base de données locale} : stockage des interactions
        \item \textbf{Système d’anonymisation} : confidentialité des échanges
    \end{itemize}
\end{frame}

% Déroulement
\section{Déroulement du projet}
\begin{frame}
    \frametitle{Déroulement du projet}
    \begin{itemize}
        \item Méthodologie agile en sprint de 2 semaines
        \item Outils utilisés :
        \begin{itemize}
            \item Trello : gestion des tâches
            \item GitHub : collaboration, versioning
            \item Streamlit : front-end
        \end{itemize}
        \item Répartition claire des rôles
    \end{itemize}
\end{frame}

% Résultats
\section{Résultats}
\begin{frame}
    \frametitle{Résultats obtenus}
    \begin{itemize}
        \item Prototype fonctionnel
        \item Tests de recommandation concluants
        \item Interface ergonomique
        \item Données utilisateurs protégées
    \end{itemize}
\end{frame}

% Conclusion
\section{Conclusion}
\begin{frame}
    \frametitle{Conclusion}
    \begin{itemize}
        \item Projet mené avec rigueur et efficacité
        \item Objectifs respectés
        \item Confidentialité au cœur du développement
        \item Enrichissement personnel et technique
    \end{itemize}
\end{frame}

% Perspectives
\section{Perspectives}
\begin{frame}
    \frametitle{Perspectives}
    \begin{itemize}
        \item Intégration d’autres types de contenus : vidéos, podcasts
        \item Optimisation de l’algorithme (auto-apprentissage)
        \item Tests utilisateurs pour améliorer l’expérience
        \item Déploiement à plus grande échelle
    \end{itemize}
\end{frame}

\begin{frame}[c]
    \centering
    \includegraphics[width=0.4\textwidth]{logo.png}

    \vspace{1cm}
    {\Huge \textbf{Merci pour votre attention !}}

    \vspace{1.5cm}
    {\Large N'hésitez pas à poser vos questions !}
\end{frame}

\end{document}